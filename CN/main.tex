%%%%%%%%%%%%%%%%%%%%%%%%%%%%%%%%%%%%%%%%%
% Medium Length Professional CV
% LaTeX Template
% Version 2.0 (8/5/13)
%
% This template has been downloaded from:
% http://www.LaTeXTemplates.com
%
% Original author:
% Trey Hunner (http://www.treyhunner.com/)
%
% Important note:
% This template requires the resume.cls file to be in the same directory as the
% .tex file. The resume.cls file provides the resume style used for structuring the
% document.
%
%%%%%%%%%%%%%%%%%%%%%%%%%%%%%%%%%%%%%%%%%

%----------------------------------------------------------------------------------------
%	PACKAGES AND OTHER DOCUMENT CONFIGURATIONS 
%----------------------------------------------------------------------------------------
 
\documentclass[UTF8, 10pt, fontset=adobe]{ctexart}
\usepackage[dvipsnames]{xcolor}
\usepackage[left=0.4 in,top=0.3 in,right=0.4 in,bottom=0.3in]{geometry} % Document margins
\usepackage{resume}

\newcommand{\tab}[1]{\hspace{.2667\textwidth}\rlap{#1}}
\newcommand{\itab}[1]{\hspace{0em}\rlap{#1}}
\name{黄天叶 (HUANG Tin Yeh)} % Your name
\address{香港理工大学工程学院工业及系统工程学系}
\address{香港特别行政区,中国,999077}
\address{+852 9449 8934 \\ +86 198 9655 5044 \\ https://tyhuang.hk}
\address{tin-yeh.huang@connect.polyu.hk \\ hty25@mails.tsinghua.edu.cn \\ huangtianye@mails.x-institute.edu.cn}

\definecolor{TsinghuaPurple}{cmyk}{0.58,0.90,0,0}
\renewenvironment{rSection}[1]{
\sectionskip
\textcolor{TsinghuaPurple}{\textbf{#1}}
\sectionlineskip
\hrule
\begin{list}{}{
%\setlength{\leftmargin}{1.5em}
\setlength{\leftmargin}{0em}
}
\item[]
}{
\end{list}
}

\begin{document}  

%----------------------------------------------------------------------------------------
%	EDUCATION SECTION
%----------------------------------------------------------------------------------------

\begin{rSection}{教育背景}
{\textbf{香港理工大学 (PolyU)}} \hfill {香港特别行政区九龙红磡}\\
{\textbf{工学学士(荣誉)} 产品工程及工业工程(双主修)} \hfill \textit{2025年9月 - 2028年5月(预期)}\\
{\textbf{工学学士(荣誉)} 产品及工业工程学} \hfill \textit{2024年9月 - 2025年8月}

{\textbf{清华大学}} (交换) \hfill {中国北京}\\
{\textbf {工学学士} 创意设计与智能工程} \hfill \textit{2025年9月 - 2026年1月(计划)}

{\textbf{香港理工大学专业进修学院 (CPCE)}} \hfill {香港特别行政区九龙油麻地}\\
{\textbf{统计及数据科学副学士}} \hfill \textit{2023年9月 - 2024年8月}
\end{rSection}

\begin{rSection}{研究学习}
{\textbf{清华大学钱学森力学班 \& X研究院}} \hfill {中国北京}\\
{\textbf{联合培养研究学者(社会与智能创新方向)}} \hfill \textit{2024年9月 - 2028年5月(预期)}\\
{导师:唐铭教授}

{\textbf{深圳医学科学院 (SMART)}} \hfill {中国广东深圳}\\
{\textbf{访问学生} - 生物分子所 (IBABI)} \hfill \textit{2025年6月 - 2025年7月}\\
{导师:胡明旭博士(PI),张琦博士}
\end{rSection}

\begin{rSection}{交流项目}
1) 上海交通大学自然科学研究院与数学科学学院,数学生物学暑期学校 \hfill \textit{2025年8月}

2)	清华大学,学期交换(2025/2026学年第一学期)	\hfill \textit{2025年9月 - 2026年1月}

3)	清华大学、X研究院、字节跳动与埃森哲,社会创新训练营	 \hfill \textit{2025年1月}

4)	布朗大学宇宙基础物理中心,AI冬季学校2025	 \hfill \textit{2025年1月}

5)	东北大学与香港理工大学,2024年教育部师生交流计划 - “人机共享控制智能车”项目 \hfill \textit{2024年12月}

6)	清华大学与X研究院,X-Challenge 2024:跨学科前沿颠覆性创新 \hfill \textit{2024年6月 -- 2024年7月}

7)	清华大学与X研究院,X-Idea | 国际暑期学校 \hfill \textit{2023年6月 -- 2023年7月}
\end{rSection}

%--------------------------------------------------------------------------------------
%   Research Publications 
%--------------------------------------------------------------------------------------
\begin{rSection}{Publications}        

1. {Wang, X.\textsuperscript{*}, \textbf{Huang, T. Y.}, Jiang, Z., \& Wang, Y. (2025). Bio-Cryptography: Dual deep learning framework for protein watermarking via geometric-chemical fingerprinting [\textbf{Poster}]. International Conference on Machine Learning (ICML) 2025 Workshop on NewInML. \\\texttt{https://icml.cc/virtual/2025/50442}} 

2. {Wang, Y.\textsuperscript{*}, Wang, J., \textbf{Huang, T. Y.}, Yang, J., Yang, G., \& Xu, Z. (2025). STGCN-LSTM for olympic medal prediction: Dynamic power modeling and causal policy optimization [\textbf{Paper}]. International Conference on Machine Learning (ICML) 2025 Workshop on NewInML. \\
\texttt{https://doi.org/10.48550/arXiv.2501.17711}} 

3. {Wang, Y.\textsuperscript{*}, Cai, M., \& \textbf{Huang, T. Y.} (2025). AI for disease prediction: Performance insights and key limitations. \textit{Journal of Clinical Neuroscience, 138}, 111360. \texttt{https://doi.org/10.1016/j.jocn.2025.111360}} 

4. {Wang, X.\textsuperscript{*}, Wang, Y.\textsuperscript{+}, \& \textbf{Huang, T. Y.\textsuperscript{+}} (2025). Crypto-ncRNA: Encryption algorithm based on non-coding RNA (ncRNA) [\textbf{Poster}]. International Conference on Learning Representations (ICLR) 2025 Workshop on AI for Nucleic Acids (NA). \texttt{https://openreview.net/forum?id=j6ODUDw4vN}} 

\textit{\footnotesize \textsuperscript{*}\,First Author \quad \textsuperscript{+}\,Co-First Author}
\end{rSection}

%--------------------------------------------------------------------------------------
%   Academic Participaction 
%--------------------------------------------------------------------------------------
\begin{rSection}{学术参与}\itemsep -3pt

{\textbf{审稿人}} \quad \textit{\textbf{ICML 2025 研讨会} · 第二届数学人工智能} \hfill \textit{2025}

{\textbf{审稿人}} \quad \textit{\textbf{F1000 Research}} \hfill \textit{2025}
\end{rSection}

%-------------------------------------------------------------------------------
%	WORK EXPERIENCE
%-------------------------------------------------------------------------------

\begin{rSection}{工作经历}
    {\textbf{AI Agent研发(兼职)}} \hfill \textit{2025年3月 -- 至今} \\
    \itab{杏仁AI有限公司}

    {\textbf{总裁助理(管理与技术总监)}} \hfill \textit{2025年2月 -- 至今} \\
    \itab{香港MBA学院}

    {\textbf{AI产品与算法研发(实习)}} \hfill \textit{2025年2月 -- 至今} \\
    \itab{清北智能科技有限公司}

    {\textbf{学生实习生}} \hfill \textit{2024年11月 -- 2024年12月} \\
    \itab{香港理工大学数据科学与人工智能研究中心 (RC-DSAI)}

    {\textbf{学生助理}} \hfill \textit{2024年10月 -- 2025年10月(预期)} \\
    \itab{香港理工大学知識轉移及創業處 (KTEO)}

    {\textbf{学生助理}} \hfill \textit{2023年3月 -- 2023年5月} \\
    \itab{香港理工大学专业进修学院科学、工程及健康学部}

    {\textbf{AI与数据培训生}} \hfill \textit{2023年12月 -- 2024年4月} \\
    \itab{帝京酒店(会计与IT部门)}\\

\end{rSection}

%-------------------------------------------------------------------------------
%	VOLUNTEERING
%-------------------------------------------------------------------------------

\begin{rSection}{志愿服务}
    {\textbf{地区青年发展及公民教育委员会委员}} \hfill \textit{2025年4月 -- 2027年4月(预期)} \\
    \itab{香港特别行政区政府民政事务总署 (HAD)}

    {\textbf{司仪 – 中华人民共和国成立75周年午餐会}} \hfill \textit{2024年10月}\\
    \itab{香港工会联合会 (HKFTU)}

    {\textbf{产品与工业工程(荣誉)工学士学生代表 (45498-PIE)}} \hfill \textit{2024年9月 – 至今}\\
    \itab{香港理工大学工业及系统工程学系 (ISE)}

    {\textbf{学生大使(社会服务时数 $>$ 30小时)}} \hfill \textit{2023年10月 -- 2024年10月} \\
    \itab{香港理工大学专业进修学院}

    {\textbf{统计学与数据科学副学士学生代表 (8C112-SDS)}} \hfill \textit{2023年9月 – 2024年8月}\\
    \itab{香港理工大学专业进修学院科学、工程及健康学部 (SEHS)}
\end{rSection}

%--------------------------------------------------------------------------------
%  Projects
%--------------------------------------------------------------------------------

\begin{rSection}{项目经历}

\begin{rSubsection}{基于多模态BERT模型的全球热浪灾害适应要素提取与分析研究} {2024年11月 - 2025年9月}{2024年中国科学院大学生创新实践训练计划}{}

\item 本研究使用多模态BERT模型整合文本、图像和结构化数据,精确识别影响热浪适应的关键因素。研究成果将为制定科学的全球热浪应对策略提供坚实的理论和数据支撑,提升社会整体灾害适应能力。
\item 导师:葛咏研究员(中国科学院地理科学与资源研究所,国家杰出青年科学基金获得者)

\end{rSubsection}  

%------------------------------------------------

\begin{rSubsection}{基于AI数字孪生技术的老年陪伴问题研究}{\textit{2024年12月 - 2025年12月}}{清华大学“一生一師一課題”學生研究訓練}{} 
\item 本项目以AI数字孪生技术为核心,旨在缓解中国人口老龄化背景下老年人情感陪伴缺失问题,同时弥合老年人数字鸿沟,提升中小学生AI实践能力。项目由深圳X研究院、字节跳动等机构联合实施,分为准备、执行、推广三个阶段。计划通过学生为老年人创建个性化数字孪生,增进代际沟通,探索AI大模型在社会公益领域的应用场景。
\item 导师:唐铭教授(国务院原参事)
\end{rSubsection}

\end{rSection}

%---------------------------------------------------------------------------------
%  Achievements
%--------------------------------------------------------------------------------

\begin{rSection}{获奖情况}
{\textbf{奖学金} - 香港特别行政区政府奖学基金——外展体验奖 (ROA)} \hfill \textit{2025}

{\textbf{奖学金} - 香港理工大学工业及系统工程学系赞助(国际会议)} \hfill \textit{2025}

{\textbf{奖学金} - 香港理工大学本科入学奖学金} \hfill \textit{2024}

{\textbf{奖学金} - 海外交流奖学金} \hfill \textit{2024}

{\textbf{奖项} - 最具学者风范奖(X研究院)} \hfill \textit{2023}

{\textbf{认证} - 微软认证:Azure AI基础} \hfill \textit{2021}

{\textbf{认证} - 数学增润课程高级证书} \hfill \textit{2021}
\end{rSection}

%----------------------------------------------------------------------------------------
%	TECHNICAL STRENGTHS SECTION
%----------------------------------------------------------------------------------------

\begin{rSection}{技能与兴趣}

\begin{tabular}{ @{} >{\bfseries}l @{\hspace{6ex}} l }
计算机  & Python (Pandas、PyTorch、TensorFlow等)、C++、SQL、SAS、R、CAD/CAE (SolidWorks)、Adobe (Photoshop、Illustrator、\\
        & InDesign)、IMOD、PyMol、ChimeraX、LaTeX、MS Office\\
\\
制造   & 增材制造 (FDM、SLA、SLS等)、逆向工程、质量工程、机器人技术 (Arduino/Raspberry Pi)、\\
        &  钳工、工程制图\\
\\
商业        & 工程经济学、管理与组织、市场营销、市场分析、工程法律\\
\\
软技能     & 产品与工业设计、产品开发、项目管理、设计思维、多元化设计 (优化、\\
        & 制造、可持续性等)、以人为本的设计\\
\\
语言       & 粤语/普通话(母语)、英语(流利)\\

\end{tabular}

\end{rSection}


%---------------------------------------------------------------------------------
%  DECLARATION
%--------------------------------------------------------------------------------

\begin{rSection}{ 声明  } \itemsep -3pt

\item 本人特此声明,上述所有信息均为本人所知所信的真实情况。

\end{rSection}
\end{document}
